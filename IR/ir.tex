\documentclass[11pt,a4paper]{jsarticle}
\usepackage[dvips]{graphicx}
\usepackage{fancyhdr}
\usepackage{here}
\setcounter{page}{0}
%
\begin{document}

\title{国際関係論レポート \\ -国際特許-}
\author{機械知能工学科 知能制御コース 3年 \\ 14104064 下松八重 宏太}
\date{提出日 \today}



\maketitle
\thispagestyle{empty}
\newpage


 \section{序論}
 国際関係論のレポートとして国際特許を選んだ理由は国際関係の話題の中でより身近な話題だと感じたからである.私は工業大学生であり,将来何かしらの研究や開発を行う上で特許問題は避けて通れない問題だ.もしも特許について知らずに自分が,または自社が開発した製品や研究した技術が先に特許を取られていた場合,特許使用料が必要になったり,特許を確認せずに使用すれば訴訟問題等になりかねない.そのため,特許について知ることは研究者にとって必須とも言える.\\
  以上よりここでは特許とその国際関係についてまとめる.
 
\section{現状}
  \subsection{特許と発明}
  さて,特許は各国の特許法で保護されている.特許法は発明を保護するための法律であるが,発明とはどのように定義されているのだろうか.特許法によると,「発明」とは「自然法則を利用した技術的思想の創作のうち高度のもの」(特許法2条1項)とされている.つまり音楽や芸術などではなく,また第三者が再現可能な創作物に適用される.最後の「高度なもの」とは実用新案法との対象の区分のためにつけられた言葉である[1].\\
   また,著作権とは違い特許は「発明」をした後に特許出願をして特許庁の審査を受けなければならない.審査に合格した「発明」を「特許発明」と呼び,これが特許法で保護される発明となる.さらに,特許法では「発明」を「物の発明」と「物を生産する方法の発明」及び「単なる方法の発明」に分類される.「物の発明」,「物を生産する方法の発明」は文字通り物品やその生産法の発明.「単なる方法の発明」とは生産物を伴わない方法,つまり通信方法などを指す.以上が日本の特許法における分類となる[2].
  
  \subsection{国際特許}
   \subsubsection{国際特許協力条約}
   国際特許協力条約(以下PCT)は1970年にワシントンで作成された.現在はWIPO(World Inttellectual Property Organization)が事務局となって活動している.日本では1978年に発効され,締約国は151ヶ国に及んでいる.PCTは発明をした人が様々な国に特許申請をしようとした時に何度も申請する手間を省くために作成された.つまり,PCTに従って国際特許申請を行うことで,WIPOがその発明を審査し,その結果と共に発明者が特許申請を行いたい締約国の国内特許庁に伝達される[3].つまり,PCTにおいては特許を保護するのはあくまでも各国の特許庁であり,PCTそのものはその発明の特許を保護するものではない.

   \subsubsection{ヨーロッパ特許付与に関する条約}
   ヨーロッパ特許付与に関する条約(以下EPC)は1973年にミュンヘンで調印された.これはPCTと同様の条約で主にヨーロッパ圏内で効力を発揮する.締約国は38ヶ国である.PCTでは国内特許と地域特許を扱っているのでPCTを利用してEPCに申請する事も出来る.
   
   \subsubsection{ヨーロッパ共同体特許条約}
   ヨーロッパ特許付与に関する条約(以下CPC)は1975年にルクセンブルクで調印された国際特許条約である.これはPCT,EPCと違い特許が各国内で保護されるのではなく,CPCで保護される.効力はEU全体に及ぶ.

 \section{問題}
 日本をはじめ多くの国が締約しているPCTではあるが,当然問題も生じている. 
  \subsection{国際出願する際の問題}
  PCTは最終的には国内の特許庁に特許申請の受理が任されている.しかし,各国の特許法における要件は様々であり,PCTが規定する要件と異なる事も多い.これによって仮にPCTを利用して国際特許申請をしたとしても,締約国によっては審査がほぼやり直しとなり,結果手間や時間が増えることとなる.これではPCTが目的としている,1つの発明に対して何度も審査や調査を行うのを避ける事が達成されない.
  
  \subsection{各国調査機関の問題}
  次に各国の特許調査機関の存在がある.PCTの利点として国内の審査の前にPCTの調査を受けることでその結果を利用することが出来ることがある.よって,国内特許庁は特許調査機関がPCTの結果を元により高水準な調査結果を報告することを期待する.しかし,アメリカなど発明が多い国の調査機関はこの点において自分たちが十分な能力を持っているので,国際調査機関を必要とせず利用する利点がないと言える.これも特許が最終的に国内特許庁によって保護されることに起因する.
  
  \subsection{発展途上国の問題}
  最後に途上国に特許申請を行うときにも問題が生じる.発展途上国においてはPCTの特許調査に関して評価出来る能力がない事もある.これは発展途上国の技術不足に起因するものである.
  
  
 \section{解決案}
 これらの問題の原因となっているのはPCTが各国の特許法をそのままに国際的な特許を保護しようとしているためである.であれば,その解決策としてはやはり明確な国際基準を規定するべきではないだろうか.CPCのように,ある国際機関がその発明の特許を保護するような枠組みを作るだと私は考える.既にWIPOという国際機関が存在しているのだからこれを利用するのが最も簡単だと考えられる.しかし,ただ国際機関が特許法を制定しただけでは国際法と国内法のどちらに優先権があるのか,特許使用についての手続きが複雑になるなどの問題が生じるだろう.これらの問題はCPCがあるEUなどの共同体ならば発生しにくいと考えられるが,世界がそのような共同体となるのはまだまだ先の事だろうと考えられる.従って,まずは国際法も制定したうえで,それを利用するかどうかは各国に任せるのが最善だろう.これによって,発展途上国などにおいても特許が十分に保護されることが期待できる.


 \section{まとめ}
国際特許においてはある程度枠組みはできているものの,国際特許として十分なものではないといえる.しかし,知的財産権といったものを国際的に保護するためには各国の協力が不可欠で,現状では真に国際特許を規定するのは難しいだろう.

\begin{thebibliography}{9}
 \bibitem[1]{1} 宮原ら,``研究者のための知的財産ハンドブック'',化学同人,2007.
 \bibitem[2]{2} 尾崎哲夫,``はじめての知的財産法'',自由国民社,2005.
 \bibitem[3]{3} 播磨良承,``国際特許法における特許協力条約'',1979.
 \bibitem[4]{4} WIPOホームページ,http://www.wipo.int/portal/en/
 \bibitem[5]{5} 特許庁ホームページ,http://www.jpo.go.jp/indexj.htm
\end{thebibliography}

\end{document}