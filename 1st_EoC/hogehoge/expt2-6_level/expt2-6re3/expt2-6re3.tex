\documentclass[11pt,a4paper]{jsarticle}
\usepackage{amsmath,amssymb}
\usepackage[dvipdfmx]{graphicx}
\usepackage{amsmath}
\usepackage{bm}
\usepackage{multirow}
\usepackage{fancyhdr}
\usepackage{listings}

\setlength{\textheight}{\paperheight}
\setlength{\topmargin}{4.6truemm} % 30mm(=1.0in+4.6mm)
\addtolength{\topmargin}{-\headheight}
\addtolength{\topmargin}{-\headsep}
\addtolength{\textheight}{-60truemm}

\setlength{\textwidth}{\paperwidth}
\setlength{\oddsidemargin}{-0.4truemm} % 25mm(=1.0in-0.4mm)
\setlength{\evensidemargin}{-0.4truemm}
\addtolength{\textwidth}{-50truemm}

\pagestyle{fancy}
\rhead{\thepage}
\lhead{}
\cfoot{}

\renewcommand{\theequation}{\arabic{section}.\arabic{equation}}
\renewcommand{\thefigure}{\thesection.\arabic{figure}}
\renewcommand{\thetable}{\thesection.\arabic{table}}
\renewcommand{\lstlistingname}{ソースコード}
\renewcommand{\headrulewidth}{0mm} % fancy

\lstset{
  basicstyle={\ttfamily \small},
  frame=trBL,
  numbers=left,
  numberstyle={\ttfamily \small},
  breaklines=true,
}


\renewcommand{\thepage}{再々\arabic{page}}

\begin{document}

\setcounter{section}{5}
\section{課題}

\subsection{測定ノイズの本実験における原因と影響}

測定ノイズの本実験における原因と影響について,
回路雑音に関する考察を追加する.

トランジスタや抵抗のような電子素子に発生する最も重要な2つの雑音に,
熱雑音とショット雑音がある.
これらの雑音は両方共システムに雑音電力を発生させ,
通常複合的な雑音効果として現れるものである.\\

熱雑音は,金属導体や抵抗内に生じる抵抗の原因である
自由電子の熱的撹乱による不規則な運動が引き起こす
導体の端の電圧ゆらぎである.

次に,ショット雑音は
アノードに向かってカソードから放出される電子によって作られる電流において,
その放出がカソードの表面状態,電極の形,
それらの間の電位に依存する不規則過程であることから,
放出される電子の数に不規則なゆらぎが生じることによるものである.\\

本実験において,以上の2種の雑音が発生していた可能性は否定できず,
発生していた場合,システムに発生した雑音電力が
直流モータに印加される電圧に影響を及ぼし,
シミュレータが表示した通りの電圧が印加されていなかった可能性が考えられる.

\begin{thebibliography}{9}
  \item F.R.コナー 原著,広田 修 訳,"電子通信工学シリーズ6 ノイズ入門",森北出版株式会社,1985,pp49-60.
\end{thebibliography}

\end{document}
