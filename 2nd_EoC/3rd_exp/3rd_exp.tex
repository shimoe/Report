\documentclass[11pt,a4paper]{jsarticle}
\usepackage[dvips]{graphicx}
\usepackage{fancyhdr}
\usepackage{here}
\setcounter{page}{0}
%
\begin{document}

\title{制御工学実験I\hspace{-0.1mm}I\hspace{-0.1mm}I \\ 熱伝導プロセスの制御}
\author{提出者 \\ 14104064 下松八重 宏太 \\ \\ 共同実験者 \\ 14101028 梶野 翔平 \\ 14104092 中島 美香 \\ 16104311 北山 拓夢}
\date{同定実験日 2016年12月21日 \\ 計算実験日 2017年1月8日 \\ 制御実験日 2017年1月8日 \\ 提出日 \today}



\maketitle
\thispagestyle{empty}
\newpage


\section{目的}
ここでは,熱交換器や強制貫流ボイラのような熱伝導プロセスの一つである加熱プロセス系を制御対象として,最短時間制御により温度を制御する方法を学ぶ.一般に熱伝導系は,長大バネの振動や建造物などにおける弾性振動と同様に,時間の他に空間を表す独立変数が必要な分布定数系として取り扱い,熱伝導系のある一点の温度に注目した大短時間制御系を構成する.

\section{実験装置の概要}
まず,実験装置は図\ref{fig1}のような構成になっている.上端がヒータによって加熱され,下端に放熱板が取り付けられている.また,温度測定用の熱電対を用いた測定システム,計算機及び変圧器から構成される.ヒータに供給される電力は変圧器によって調整可能であり,計算機によって測定温度の記録を行う. \\
次に,加熱プロセスの詳細を図\ref{fig2}に示す.加熱プロセスは銅棒の表面から熱が拡散されるのを抑えるため,保温用の綿が巻きつけられている.銅棒は上端から加熱され,放熱板につながる下端から冷却される.銅棒の各点の温度はシース型熱電対によって固定され,A/D変換器を経て計算機に記録される.

\section{原理}
  \subsection{制御対象の伝達関数}
  $x$[m]を銅棒におけるヒータからの距離,$t$[s]を時間,$\theta(x,t)$をその時の温度とする.初期条件$\theta_x(x,0) = 0$つまり温度勾配が$0$で,ステップ状の入力$Q_0(t) = Q_0$[J]が印加された場合,この系は次式で表現出来る事が知られている.
  \begin{equation}
   \theta(x,t) = \sum_{n = 1}^{\infty} 
    \frac{2}{\pi + 2b_n} 
    \frac{Q'_0 \sin(a_n x)}{x a_n^3} 
    \cos \{ a_n x(1-e^{-a_n^2 t}) \}
  \end{equation}
ここで,$a_n$は$\tan(a_n \pi)$
\section{実験方法}
\section{結果}
\section{考察}

\end{document}