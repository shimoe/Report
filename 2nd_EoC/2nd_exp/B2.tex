\documentclass[11pt,a4paper]{jsarticle}
\usepackage[dvips]{graphicx}
\usepackage{fancyhdr}
\usepackage{here}
\setcounter{page}{0}
%
\begin{document}

\title{制御工学実験I \hspace{-0.1mm} I \hspace{-0.1mm} I \\ B2.台車の位置制御}
\author{提出者 \\ 14104064 下松八重 宏太 \\ \\ 共同実験者 \\ 14101028 梶野 翔平 \\ 14104092 中島 美香 \\ 16104311 北山 拓夢}
\date{同定実験日 2016年11月21日 \\ 計算実験日 2016年11月28日 \\ 制御実験
日 2016年12月5日 \\ 提出日 \today}



\maketitle
\thispagestyle{empty}
\newpage


\section{目的}
状態フィードバック制御を行うには全状態量をフィードバックする必要がある.
しかしセンサの不足などによって実際に測定できない状態量が存在する場合には,
オブザーバを用いて状態量を推定し,フィードバック制御に利用する.本実験で
は台車の位置制御実験を通じて,オブザーバによる状態フィードバック制御につ
いて理解を深める.
\section{原理}
  \subsection{数式モデル}
  モータの回転による力$T_u$を入力,台車の変位$y$を出力としたとき,運動方程式は次のようになる.
  \begin{equation}
   M\ddot{y}+F\ddot{y}=T_u
  \end{equation}
  $M$は台車の透過質量,$F$はレールと台車の間の摩擦係数である.$T_u$はパソコンからの指令電圧$u$に比例する.すなわち,$T_u = \alpha u$.ただし,$\alpha = 20.09$[N/V]. \\
  状態変数を$x = [x_1 x_2]^T = [y \dot{y}]$とおくと,次の状態方程式が得られる.
  \begin{eqnarray}
   \dot{x} &=& \left[
		\begin{array}{cc}
		 0 & 1 \\
		 0 & -\frac{F}{M}\\
		\end{array}
		  \right]
   + \left[
      \begin{array}{c}
       0 \\
       \frac{\alpha}{M}\\
      \end{array}
     \right]u \nonumber \\
   y &=& [1 \ 0]x
  \end{eqnarray}

  \subsection{未知パラメータの同定}
  状態フィードンバック制御系を構成するためには,式2の動特性パラメータがわかっている必要がある.システムのステップ応答より$F$と$M$を同定することが出来る.式2より,台車に一定の力を入力すると,台車は1方向に走り続ける.そこで,次の出力フィードバックをかける.
  \begin{equation}
   u = h(r-y)
  \end{equation}
  ここで,$r$は台車の目標変位,$h$はフィードバックゲイン定数である.すると,$r$から$y$までの伝達関数は次のようになる.また,この系の応答波形は$h$の値によって変わる.
  \begin{equation}
 G(s) = \frac{Y(s)}{R(s)} = \frac{b_0}{s^2+a_1 s +a_0} \ \ (a_1 = F/M,\ a_0 = b_0 = \alpha h/M)
  \end{equation}
\section{実験方法}

\section{結果}

\section{考察}

\end{document}